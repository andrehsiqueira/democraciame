%!TEX root = main.tex
%%% Texto do artigo
%% ================================
%% TEXTO DO ARTIGO
%% ================================

%% \section*{ Sessão 1}


A democracia é um modelo político no qual prevalece a decisão da maioria. Numa forma simples, a maioria decide o que é melhor para todos. Surge como uma estrutura de decisão que contempla a vontade do homem livre, que deixa de submeter-se às arbitrariedades de tiranos, imperadores ou reis absolutistas. O desenho da democracia pode ser encontrado em alguns documentos relevante como a Oração Fúnebre de Péricles (431 a.C.)~ \cite{oracaoPericles}, a Declaração de Independência dos Estados Unidos da América \cite{euadeclara}, a Declaração dos Direitos do Homem e do Cidadão \cite{dcidadao} e o Discurso de Lincoln em Gettysburg \cite{dlincoln}. Todos eles são fontes determinantes para compreender a natureza da democracia e identificar  os três elementos básicos que caracterizam a democracia: \textbf{participação, igualdade, e liberdade}.

Muitas ações de melhoria no processo das decisões democráticas resultam em um amadurecimento na arquitetura das decisões, tornando a democracia a forma de decisão política mais adequada ao estado de liberdade, mesmo com suas eventuais dificuldades. Um dos problemas que restam à democracia é a sua dependência em relação à informação. Só existe decisão livre quando há escolhas e só há escolhas acertadas quando há informações. Sendo a democracia dependente de escolhas livres, o acesso a informações condiciona a liberdade de decidir. Se há manipulação de informações surge a manipulação das escolhas. A democracia sofre o seu revés quando o homem, julgando-se livre para fazer escolhas, não é livre para analisar as escolhas que tem. 
Ainda no tema da liberdade, a imposição da vontade da maioria, mesmo quando benéfica, restringe a liberdade do indivíduo ao impor a ele a obrigatoriedade de atender a decisão que lhe é contrária aos interesses. Pior ainda quando a decisão é manipulada, pois seque atende aos interesses de que escolheu a decisão, que nem sempre resulta na melhor alternativa para todos. 

A Teoria da Escolha Pública e a Teoria da Escolha Social apresentam-se como esquemas referenciais para analisar a estrutura da democracia. Em ambas o problema das escolhas coletivas aparece como ponto de foco:

\begin{citacao}
	Do ponto de vista individual, a escolha já representa um desafio psicológico de avaliação dos custos, benefícios e riscos da decisão. Quando a decisão é coletiva, ela depende da decisão dos outros indivíduos do grupo. Nesse sentido, é possível verificar, com somente alguns conceitos da Teoria da Escolha Social, que a agregação de preferências individuais conflitantes, na tentativa de extrair uma escolha única em busca do bem-estar coletivo, está longe de ser trivial, está envolta de paradoxos e não aparenta ter uma solução ótima ou unânime. A agregação fica ainda mais complexa se for considerada a variável tempo, ou seja, como a vontade do eleitor se comporta ao longo de influências diversas até que haja as eleições ou o processo de escolha. Os desafios de extrair a vontade coletiva não residem somente na regra de agregação mais adequada, mas também em como preservar a vontade do eleitor contra influências ilegítimas e se esta vontade está representada no cômputo final. \cite[p. 45]{rafaelazevedo2014}
\end{citacao}

Os grupos sociais deparam-se com dificuldades de escolhas: são os problemas da democracia. É preciso proporcionar mecanismos de participação e respeitar diferenças; reconhecer a igualdade de direitos e, ao mesmo tempo, preservar a liberdade. Não é uma equação de equilíbrio simples. Por isso as dificuldades democráticas afetam as diferentes organizações: nações, estados, instituições e famílias. Neste conjunto de estruturas sociais a enfrentar os desafios democráticos desponta naturalmente o movimento espírita.

Como expressão de democracia, o movimento espírita é um esforço organizado para reunir pessoas e recursos, de maneira estruturada, para promover o estudo, a vivência e a divulgação do Espiritismo. A organização do movimento espírita brasileiro é de natureza federativa e foi desenhada através de um acordo voluntário \textit{ad referendum} que entre si fizeram alguns representantes espíritas quando em 1949 assinaram conjuntamente o documento denominado Pacto Áureo\cite{pactoaureo2012}, como forma de promover a participação, respeitar a igualdade e a liberdade dos agentes do movimento espírita como grupo social: indivíduos e instituições. 

Do acordo firmado pelo Pacto Áureo resultou a criação do Conselho Federativo Nacional (CFN) na Federação Espírita Brasileira (FEB), um colegiado composto por representantes das Federativas estaduais. Assim como a FEB mantém o CFN, cada Federativa estadual é convidada a manter o Conselhos Federativo Estadual. O papel dos Conselhos Federativos Estaduais é tratar das ações e projetos do movimento espírito em âmbito estadual assim como, similarmente, o Conselho Federativo Nacional trata do contexto nacional. Cada instituição espírita do estado tem direito a voz e voto nos Conselhos Federativos Estaduais, assim como cada estado tem direito a voz e voto no Conselho Federativo Nacional. Por meio desta organização busca-se resolver a equação de equilíbrio entre participação, igualdade e liberdade no campo do movimento espírita brasileiro.

Merece especial destaque o cuidado do Pacto Áureo na preservação da participação, da igualdade e da liberdade institucional de seus componentes quando, no item 12, temos que:


\begin{citacao}
As Sociedades \textbf{componentes} do Conselho Federativo Nacional são \textbf{completamente independentes}. A ação do Conselho só se verificará, aliás, \textbf{fraternalmente}, no caso de alguma Sociedade passar a adotar programa que colida com a doutrina exposta nas obras: “O Livro dos Espíritos” e “O Livro dos Médiuns”, e isso por ser ele, o Conselho, o orientador do Espiritismo no Brasil (\emph{grifos nossos}).
\end{citacao}

A participação é facultada ao representante estadual que congrega o maior número de instituições espíritas do estado ou distrito federal. A natureza fraterna de suas relações definem uma condição de completa igualdade, sem qualquer caráter de subordinação e a completa independência lhes garante o direito de liberdade. Engana-se quem julgar que a estrutura do CFN possui uma hierarquia sobre sociedades espíritas ou indivíduos. Compreendendo que a liberdade é um dos fundamentos do pensamento espírita entende-se que não pode haver progresso sem o natural exercício da liberdade.  

O Conselho Federativo Nacional é um fórum democrático para a apreciação de temas de interesse do movimento espírita brasileiro e para planejar e executar ações coletivas de estudo, vivência e divulgação do espiritismo sobre as bases da codificação kardequiana e promovendo a participação, a igualdade e a liberdade de seus membros. 

Na condição de orientador do movimento espirita brasileiro -- que neste contexto significa dar orientação --  não oferece ao CFN qualquer tipo de supremacia; trata-se de um colegiado voluntário com vistas a discutir e coordenar as melhores formas de promover o estudo, a vivência e a divulgação do espiritismo. Sua função é oferecer orientações de trabalho que resultam da apreciação e debate de temas entre os membros participantes, não é subordinar vontades ou impor decisões às instituições espíritas brasileiras. As orientações do CFN se expressam na forma de documentos emitidos que servem como auxiliares para o estudo, a prática e a vivência do espiritismo, sem que tenham -- destaque-se -- qualquer caráter de obrigatoriedade, uma vez que sua adoção é sempre facultativa. Justo salientar que a Federação Espírita Brasileira (FEB), é a entidade que preside o CFN, mas nele não tem voto! A FEB é responsável por coordenar as ações definidas pelas federativas que compõe o CFN. É também a sociedade responsável por reunir os representantes federativos do movimento espírita brasileiro para coletar e promover a análise e o debate de assuntos que interessem ao movimento espírita brasileiro. Decorrente das deliberações, a FEB é incumbida de coordenar as ações e projetos em âmbito nacional, através da colaboração voluntária das instituições, sempre respeitando a autonomia das sociedades e a liberdade de escolha dos indivíduos. Destacado o caráter \emph{ad referendum} de sua proposta - que significa ``sujeito à aceitação posterior por parte de um colegiado (diz-se de ato tomado isoladamente)'', é necessário compreendermos que o Pacto Áureo não estabeleceu uma hierarquia à qual se deva obediência, mas estabeleceu uma proposta de um programa de trabalho democraticamente constituído e ao qual podemos, ou não, aderir conforme nossas convicções e interesses em relação ao movimento espírita brasileiro.   

Como doutrina libertária da alma, o Espiritismo evoca a necessidade do progresso intelectual e moral \cite[Parte III. Cap. VII ]{Kardec1857} e evoca a Lei de Liberdade \cite[Parte III. Cap.X]{Kardec1857}, como fundamento para as construções do Espírito imortal! Progresso e Liberdade figuram na Filosofia Espírita como componentes das leis morais que regulam o universo espiritual com a mesma naturalidade com que as leis físicas regulam o universo da matéria. Assim o Espiritismo é progressista e libertário em sua natureza. Compreende o Espírito como ente natural e sujeito ao contínuo aprendizado das leis divinas através das múltiplas experiências que o Universo lhe proporciona. Todos os seres estão em contínua interação e o resultado do aprendizado das leis naturais é o progresso -- aqui entendido como a contínua aproximação de um estado de melhoria em concordância com as Leis Divinas (que são as leis naturais). 

Compreendendo as dificuldades de aprendizado das leis naturais e vacinado contra o aprisionamento intelectual de uma postura absolutista, escreveu \citeonline[p. 42]{Kardec1869}:

\begin{citacao}
	Caminhando de par com o progresso, o Espiritismo jamais será ultrapassado, porque, se novas descobertas lhe demonstrassem estar em erro acerca de um ponto qualquer, ele se modificaria nesse ponto. Se uma verdade nova se revelar, ele a aceitará.

\end{citacao}

No atual estágio de progresso, o movimento espírita utiliza o processo democrático representativo para analisar temas e para tomar decisões no âmbito do CFN. É natural que ocorram erros nas decisões tomadas no atual estágio de desenvolvimento da nossa sociedade. Mas devemos identificar e praticar o caráter progressivo do movimento espírita que deve esforçar-se por identificar erros acerca deste ou daquele ponto e modificar os conceitos ou as decisões acerca daquele ponto através da participação de todos no trabalho, da igualdade solidária fraterna e da tolerância que advém da liberdade. É assim que as instituições progridem, como os indivíduos. Aos conceitos de participação, igualdade e liberdade, o movimento espírita deve aduzir as noções de Trabalho, Solidariedade e Tolerância.
%% TODO> Incluir citação!

Ao respeitar o livre-arbítrio das sociedades e dos indivíduos, o movimento espírita evita a ditadura da democracia, cuja imposição da vontade da maioria, pode obscurecer o direito de liberdade do indivíduo. Esta é uma atitude de tolerância. Compreendendo que cada sociedade espírita deve encontrar o seu caminho de progresso, o movimento espírita recomenda, orienta e informa, sem impor, obrigar ou exigir -- isto expressa solidariedade. Reunidos em torno do trabalho que nos une, a democracia se exprime com autonomia, participação e liberdade.
 

É sempre oportuno analisar o valor da estrutura democrática do movimento espírita e o quanto, voluntaria e autonomamente, podemos ganhar com a adoção de suas propostas \footnote{Especialmente neste ano de 2019, quando se comemoram os 70 anos da assinatura do Pacto Áureo.}. Desde sua constituição, o movimento espírita é um coletivo social que aprende, erra, corrige e se aperfeiçoa. Ao longo dos anos muitas modificações foram propostas para o seu funcionamento mas sua estrutura permanece essencialmente democrática. Vigem o continuo esforço de promover a participação; a persistência fraterna que garante a igualdade; e o completo respeito à autonomia das sociedades que enaltece a liberdade dos indivíduos. Juntos estes caracteres mantêm o compromisso com o progresso dos espíritos e com a melhoria da sociedade. O movimento espírita aparece assim como uma expressão democrática a reunir a coletividade dos voluntários que trabalham nas instituições espíritas para promover o estudo, a vivência e a divulgação do Espiritismo. Sua estruturação é dinâmica e atende aos preceitos da participação, da autonomia e da liberdade, guardando concordância com o caráter de desenvolvimento de trabalhos coletivos vivenciados por Allan Kardec: Trabalho, Tolerância e Solidariedade \cite[p. 14]{Kardec1890}. É preciso compreender que ninguém é obrigado a dele participar e que o nosso principal objetivo é a melhoria de nós mesmo, como afirma o Espírito Emmanuel \cite[Edição Kindle: comentário a João 6:32]{emmEvJoao}:

\begin{citacao}
Toda arregimentação religiosa na Terra não tem escopo maior que o de preparar as almas, ante a grandeza da vida espiritual.\\
Templos de pedra arruínam-se.\\
Princípios dogmáticos desaparecem.\\
Cultos externos modificam-se.\\
Revelações ampliam-se.\\
Sacerdotes passam.\\
Todos os serviços da fé viva representam, de algum modo, aquele pão que Moisés dispensou aos hebreus, alimento valioso sem dúvida, mas que sustentava o corpo apenas por um dia, e cuja finalidade primordial é a de manter a sublime oportunidade da alma em busca do verdadeiro pão do Céu.
\end{citacao}
 
 Deste modo, recordemos de que nossos esforços de promover a participação, a igualdade e a liberdade -- dísticos notáveis da democracia, representam  oportunidades de cooperação social, mas o nosso maior e mais importante compromisso continua sendo a transformação de nossas almas em busca da iluminação de nós mesmos.

\vspace{1,2cm}
\hspace{\fill} Brasília, Janeiro de 2019.