%!TEX root = main.tex
%%% Texto do artigo
%% ================================
%% TEXTO DO ARTIGO
%% ================================

%% \section*{ Sessão 1}


A democracia é um modelo político no qual prevalece a decisão da maioria. Numa forma simples, a maioria decide o que é melhor para todos. Surge como uma estrutura de decisão que contempla a vontade do homem livre, que deixa de submeter-se às arbitrariedades de tiranos, imperadores ou reis absolutistas. O desenho da democracia pode ser encontrada em alguns documentos relevante: a Oração Fúnebre de Péricles (431 a.C.)~ \cite{oracaoPericles}, o Acordo do Povo - uma série de manifestos publicados entre 1647 a 1649 para mudanças na Inglaterra, a declaração de independência dos Estados Unidos da América, a Declaração dos Direitos do Homem e do Cidadão e o discurso de Lincoln em Gettysburg são fontes importantes para compreendermos a natureza da democracia. Nestes documentos, podemos identificar  os três elementos básicos apresentados por eles como características de uma democracia: participação, liberdade e igualdade.

Muitas tentativas de melhoria no processo das decisões democráticas resultam em um amadurecimento na arquitetura das decisões, tornando a democracia a forma de decisão política mais adequada ao estado de liberdade, mesmo com suas eventuais dificuldades. Uma das dificuldades da democracia é a sua dependência da informação. Só existe decisão quando há escolhas e só há escolhas quando há informação. Como a democracia depende das escolhas, depende indiretamente do acesso a informação. Também ocorre que quando há manipulação de informações há manipulação de escolhas. E a democracia sofre o seu revés quando o homem, julgando-se livre para fazer escolhas, não é livre para analisar as escolhas que tem. 
Outro problema relativo à democracia é a submissão de todos à decisão da maioria. Em um dissertação de Mestrado, Rafael \citeonline[p. 45]{rafaelazevedo2014} analisa duas teorias sobre escolhas (A Teoria da Escolha Pública e a Teoria da Escolha Social) para destacar o problema das escolhas democráticas, como referenciado e destacando:

\begin{citacao}
	Do ponto de vista individual, a escolha já representa um desafio psicológico de avaliação dos custos, benefícios e riscos da decisão. Quando a decisão é coletiva, ela depende da decisão dos outros indivíduos do grupo. Nesse sentido, é possível verificar, com somente alguns conceitos da Teoria da Escolha Social, que a agregação de preferências individuais conflitantes, na tentativa de extrair uma escolha única em busca do bem-estar coletivo, está longe de ser trivial, está envolta de paradoxos e não aparenta ter uma solução ótima ou unânime. A agregação fica ainda mais complexa se for considerada a variável tempo, ou seja, como a vontade do eleitor se comporta ao longo de influências diversas até que haja as eleições ou o processo de escolha. Os desafios de extrair a vontade coletiva não residem somente na regra de agregação mais adequada, mas também em como preservar a vontade do eleitor contra influências ilegítimas e se esta vontade está representada no cômputo final.
\end{citacao}



Por sua vez, o movimento espírita é um esforço organizado para reunir pessoas e recursos, de maneira estruturada, para promover o estudo, a vivência e a divulgação do Espiritismo. A estrutura do movimento espírita brasileiro é de natureza federativa e foi desenhada através de um acordo voluntário \textit{ad referendum} que entre si fizeram alguns representantes espíritas quando em 1949 assinaram conjuntamente o documento denominado Pacto Áureo\cite{pactoaureo2012}, do qual resultou a criação do Conselho Federativo Nacional (CFN). 

Apesar das críticas existentes sobre o documento e sua origem, identificamos nele um desenho da estrutura do Movimento Espírita Brasileiro. Em termos gerais, ele define um acordo entre entidades que representariam os movimentos espíritas estaduais pela adesão aos preceitos do Pacto. Merece especial destaque o cuidado na preservação da liberdade institucional de seus componentes quando, no item 12, temos que:


\begin{citacao}
As Sociedades componentes do Conselho Federativo Nacional são \textbf{completamente independentes} (\emph{grifos nossos}). A ação do Conselho só se verificará, aliás, fraternalmente, no caso de alguma Sociedade passar a adotar programa que colida com a doutrina exposta nas obras: “O Livro dos Espíritos” e “O Livro dos Médiuns”, e isso por ser ele, o Conselho, o orientador do Espiritismo no Brasil.
\end{citacao}

A natureza independente das sociedades que compõe o CFN e a natureza fraterna de suas relações definem uma condição de completa igualdade, sem qualquer caráter de subordinação. 
Engana-se quem julgar que a estrutura do CFN possui qualquer hierarquia sobre as sociedades espíritas ou sobre qualquer indivíduo. Além de compreender que a lei de liberdade é um dos fundamentos do pensamento espírita que não pode haver progresso sem o natural exercício da liberdade, o CFN não caracteriza-se como uma instância hierárquica para o movimento espírita brasileiro, mas como um fórum democrático para a apreciação de temas de interesse do movimento espírita brasileiro e para planejar e executar ações coletivas de estudo, vivência e divulgação do espiritismo sobre as bases da codificação kardequiana. Ainda do item 12, destaquemos a condição de orientador do CFN. Tal condição não lhe dá qualquer tipo de primazia ou poder. Sua função são é subordinar instituições, mas oferecer orientações de trabalho que resultam da apreciação e debate de temas entre os membros participantes. Tais orientações se expressam na forma de documentos emitidos pelo CFN como auxiliares para o estudo, a prática e a vivência do espiritismo, sem que tenham -- destaque-se -- qualquer caráter de obrigatoriedade.

Não há Espiritismo sem o respeito à Lei de Progresso \cite[Parte III. Cap. VII ]{Kardec1857} e à Lei do Liberdade \cite[Parte III. Cap.X]{Kardec1857}, ambas caracterizadas na Filosofia Espírita como partes das leis morais que regulam o universo do espírito como as leis físicas regulam o universo da matéria. Portanto, o Espiritismo é progressista em sua natureza. Ao definir o espírito como ente natural e sujeito ao contínuo aprendizado das leis divinas através das múltiplas experiências que o Universo lhe proporciona, a doutrina espírita assinala que todos os seres estão em contínua interação, cujo propósito é o progresso -- aqui entendido como a contínua aproximação de um estado de melhoria em concordância com as Leis Divinas (que são as leis naturais). Sobre a progressividade da Doutrina Espírita, escreveu \citeonline[p. 42]{Kardec1869}

\begin{citacao}
	Caminhando de par com o progresso, o Espiritismo jamais será ultrapassado, porque, se novas descobertas lhe demonstrassem estar em erro acerca de um ponto qualquer, ele se modificaria nesse ponto. Se uma verdade nova se revelar, ele a aceitará.

\end{citacao}

No atual estágio de progresso, o movimento espírita utiliza o processo democrático representativo para analisar temas e para tomar decisões no âmbito do CFN. Nele todos os estados estão representados por suas federativas, que por sua vez intentam representar as instituições espíritas de suas localidades. Mas as decisões tomadas não têm caráter impositivo. Tratam-se de recomendações ou orientações cuja adesão é sempre voluntária, sempre baseada no respeito à liberdade e autonomia de cada sociedade ou indivíduo. Neste aspecto o movimento espírita não impõe as decisões com base na vontade da maioria. Planeja, analisa e orienta conforme a decisão da maioria, mas a adoção desta ou daquela plataforma, deste ou daquele modelo de ação, desta ou daquela atitude é sempre uma decisão da sociedade espírita ou do indivíduo, como recomenda a lição de autonomia exarada no Pacto Áureo, em seu item 12. É natural que ocorram erros nas decisões tomadas pelos homens ou instituições no atual estágio de desenvolvimento da nossa sociedade. Mas aqui também devemos adotar o caráter progressivo do movimento espírita -- inspirado na própria doutrina espírita -- que deve ao identificar erros acerca de determinados pontos, modificar conceitos ou decisões acerca daquele ponto. É assim que as instituições progridem, como os indivíduos. 

Ao respeitar o livre-arbítrio das sociedades e dos indivíduos, o movimento espírita evita a ditadura da democracia, cuja imposição da vontade da maioria, pode obscurecer o direito de liberdade do indivíduo. Compreendendo que cada sociedade espírita deve encontrar o seu caminho de progresso, o movimento espírita recomenda, orienta e informa, sem impor, obrigar ou exigir. 

Por esta natureza "ultra"\ democrática do movimento espírita, não se pode esperar responsabilidades hierárquicas. O Conselho Federativo Nacional não é o responsável pelas ações dos espíritas brasileiros, dada a completa autonomia de cada sociedade e de cada indivíduo. 

Justo salientar que a Federação Espírita Brasileira (FEB), é a entidade que preside o CFN, mas nele não tem voto! A FEB é responsável por coordenar a as ações definidas pelas federativas que compõe o CFN. É também a sociedade responsável por reunir os representantes federativos do movimento espírita brasileiro para coletar e promover a análise e o debate de assuntos que interessem ao movimento espírita brasileiro. Decorrente das deliberações, a FEB é incumbida de coordenar as ações e projetos em âmbito nacional, através da colaboração voluntária das instituições, sempre respeitando a autonomia das sociedades e a liberdade de escolha dos indivíduos. 

Neste ano comemoram-se os 70 anos do Pacto Áureo. É oportuno analisar o valor da estrutura desenvolvida ao longo destes anos e o quanto, voluntaria e autonomamente, podemos ganhar com a adoção de suas propostas. Desde a sua constituição, modificações foram propostas para o funcionamento do Conselho Federativo Nacional, que se refletem em seu Regimento Interno, mas a essência democrática de sua estrutura, o completo respeito à autonomia das sociedades, tanto quanto  à liberdade dos indivíduos,  mantêm o compromisso com o progresso dos espíritos e com a melhoria da sociedade.
 
Destacado o caráter \emph{ad referendum} de sua proposta - que significa ``sujeito à aceitação posterior por parte de um colegiado (diz-se de ato tomado isoladamente)'', é necessário compreendermos que o Pacto Áureo não estabeleceu uma hierarquia à qual se deva obediência, mas estabeleceu uma proposta de um programa de trabalho democraticamente constituído e ao qual podemos, ou não, aderir conforme nossas convicções e interesses em relação ao movimento espírita brasileiro. Sua proposta continua em concordância com o caráter de desenvolvimento de trabalhos coletivos proposto por Allan Kardec: Trabalho, Tolerância e Solidariedade. E qualificamos: democrática, progressiva e voluntariamente.  

\vspace{1,2cm}
\hspace{\fill} Brasília, Janeiro de 2019.