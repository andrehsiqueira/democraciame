%% main.tex  
%% Tema: A transversalidade da Comunicação Social Espírita  

%% ----------------------------------------------------------------

\documentclass[
	% -- opções da classe memoir --
	article,			% indica que é um artigo acadêmico
	11pt,				% tamanho da fonte
	oneside,			% para impressão apenas no verso. Oposto a twoside
	a4paper,			% tamanho do papel. 
	% -- opções da classe abntex2 --
	%chapter=TITLE,		% títulos de capítulos convertidos em letras maiúsculas
	%section=TITLE,		% títulos de seções convertidos em letras maiúsculas
	%subsection=TITLE,	% títulos de subseções convertidos em letras maiúsculas
	%subsubsection=TITLE % títulos de subsubseções convertidos em letras maiúsculas
	% -- opções do pacote babel --
	english,			% idioma adicional para hifenização
	brazil,				% o último idioma é o principal do documento
	]{abntex2}


% ---
% PACOTES
% ---

% ---
% Pacotes fundamentais 
% ---
\usepackage{cmap}				% Mapear caracteres especiais no PDF
%\usepackage{lmodern}			% Usa a fonte Latin Modern
\usepackage[sfdefault]{roboto} 
\usepackage[T1]{fontenc}		% Selecao de codigos de fonte.
\usepackage[utf8]{inputenc}		% Codificacao do documento (conversão automática dos acentos)
\usepackage{indentfirst}		% Indenta o primeiro parágrafo de cada seção.
\usepackage{nomencl} 			% Lista de simbolos
\usepackage{color}				% Controle das cores
\usepackage{graphicx}			% Inclusão de gráficos
% ---
		
% ---
% Pacotes adicionais, usados apenas no âmbito do Modelo Canônico do abnteX2
% ---
 \usepackage{lipsum}				% para geração de dummy text
% ---
		
% ---
% Pacotes de citações
% ---
% \usepackage[brazilian,hyperpageref] {backref}	 % Paginas com as citações na bibl
\usepackage[alf,abnt-full-initials=yes]{abntex2cite}	% Citações padrão ABNT
% ---

\usepackage{lineno}
%\linenumbers %% Exibe numero de linhas



% ---
% Configurações do pacote backref
% Usado sem a opção hyperpageref de backref
% \renewcommand{\backrefpagesname}{Citado na(s) página(s):~}
% % Texto padrão antes do número das páginas
% \renewcommand{\backref}{}
% % Define os textos da citação
% \renewcommand*{\backrefalt}[4]{
% 	\ifcase #1 %
% 		Nenhuma citação no texto.%
% 	\or
% 		Citado na página #2.%
% 	\else
% 		Citado #1 vezes nas páginas #2.%
% 	\fi}%
% % ---

% ---
% Informações de dados para CAPA e FOLHA DE ROSTO
% ---
% \titulo{Modelo Canônico de\\ Artigo científico com \abnTeX}
% \autor{Equipe \abnTeX\thanks{\url{http://abntex2.googlecode.com/}} \and Lauro
% César
% Araujo\thanks{laurocesar@laurocesar.com}}
% \local{Brasil}
% \data{2013, v-1.7.1}
% ---

% ---
% Configurações de aparência do PDF final

% alterando o aspecto da cor azul
\definecolor{black}{RGB}{41,5,195}

% informações do PDF
\makeatletter
\hypersetup{
     	%pagebackref=true,
		pdftitle={\@title}, 
		pdfauthor={\@author},
    	pdfsubject={\@title},
	    pdfcreator={LaTeX with abnTeX2},
		pdfkeywords={Comunicação Social Espírita}{Espiritismo}, 
		colorlinks=true,       		% false: boxed links; true: colored links
    	linkcolor=black,          	% color of internal links
    	citecolor=black,        		% color of links to bibliography
    	filecolor=magenta,      		% color of file links
		urlcolor=black,
		bookmarksdepth=4
}
\makeatother
% --- 

% ---
% compila o indice
% ---
\makeindex
% ---

% ---
% Altera as margens padrões
% ---
\setlrmarginsandblock{3cm}{3cm}{*}
\setulmarginsandblock{3cm}{3cm}{*}
\checkandfixthelayout
% ---

% --- 
% Espaçamentos entre linhas e parágrafos 
% --- 

% O tamanho do parágrafo é dado por:
\setlength{\parindent}{1,25cm}

% Controle do espaçamento entre um parágrafo e outro:
\setlength{\parskip}{1 em}  % tente também \onelineskip ou 0.2cm

% Espaçamento simples
\SingleSpacing

% ----
% Início do documento
% ----
\begin{document}

% Retira espaço extra obsoleto entre as frases.
\frenchspacing 



%% CONFIGURAÇÕES
%% =============
 \linenumbers %% Exibe numero de linhas
%% \modulolinenumbers[2] %% exibe a cada [x] linhas

\citeoption{abnt-repeated-author-omit=yes}

%% ================================
%% DADOS
%% ================================
\titulo{Movimento Espírita e Democracia}
\autor{Andr\'{e} Henrique de Siqueira}
\data{Janeiro de 2019}

%% ================================
%% TITULO
%% ================================

\begin{center}
   \textbf{\fontsize{20}{30}\selectfont \MakeUppercase \imprimirtitulo}
\end{center}
\vspace{1,3 cm}
{\hspace*{\fill} \footnotesize \imprimirautor}\\
{\hspace*{\fill} {\fontsize{8}{9}\selectfont andrehsiqueira@febnet.org.br}}

\vspace{1,5 cm}



% ----------------------------------------------------------
% TEXTO DO ARTIGO
% ----------------------------------------------------------
%!TEX root = main.tex
%%% Texto do artigo
%% ================================
%% TEXTO DO ARTIGO
%% ================================

%% \section*{ Sessão 1}


A democracia é um modelo político no qual prevalece a decisão da maioria. Numa forma simples, a maioria decide o que é melhor para todos. Surge como uma estrutura de decisão que contempla a vontade do homem livre, que deixa de submeter-se às arbitrariedades de tiranos, imperadores ou reis absolutistas. A oração fúnebre de Péricles (431 a.C.), o Acordo do Povo - uma série de manifestos publicados entre 1647 a 1649 para mudanças na Inglaterra, a declaração de independência dos Estados Unidos da América, a Declaração dos Direitos do Homem e do Cidadão e o discurso de Lincoln em Gettysburg são fontes importantes para compreendermos a natureza da democracia. Nestes documentos, podemos identificar  os três elementos básicos apresentados por eles como características de uma democracia: participação, liberdade e igualdade.

Muitas tentativas de melhoria no processo das decisões democráticas resultam em um amadurecimento na arquitetura das decisões, tornando a democracia a forma de decisão política mais adequada ao estado de liberdade, mesmo com suas eventuais dificuldades. Uma das dificuldades da democracia é a sua dependência da informação. Só existe decisão quando há escolhas e só há escolhas quando há informação. Como a democracia depende das escolhas, depende indiretamente do acesso a informação. Também ocorre que quando há manipulação de informações há manipulação de escolhas. E a democracia sofre o seu revés quando o homem, julgando-se livre para fazer escolhas, não é livre para analisar as escolhas que tem. 
Outro problema relativo à democracia é a submissão de todos à decisão da maioria. Em um dissertação de Mestrado, Rafael \citeonline[p. 45]{rafaelazevedo2014} analisa duas teorias sobre escolhas (A Teoria da Escolha Pública e a Teoria da Escolha Social) para destacar o problema das escolhas democráticas, como referenciado e destacando:

\begin{citacao}
	Do ponto de vista individual, a escolha já representa um desafio psicológico de avaliação dos custos, benefícios e riscos da decisão. Quando a decisão é coletiva, ela depende da decisão dos outros indivíduos do grupo. Nesse sentido, é possível verificar, com somente alguns conceitos da Teoria da Escolha Social, que a agregação de preferências individuais conflitantes, na tentativa de extrair uma escolha única em busca do bem-estar coletivo, está longe de ser trivial, está envolta de paradoxos e não aparenta ter uma solução ótima ou unânime. A agregação fica ainda mais complexa se for considerada a variável tempo, ou seja, como a vontade do eleitor se comporta ao longo de influências diversas até que haja as eleições ou o processo de escolha. Os desafios de extrair a vontade coletiva não residem somente na regra de agregação mais adequada, mas também em como preservar a vontade do eleitor contra influências ilegítimas e se esta vontade está representada no cômputo final.
\end{citacao}



Por sua vez, o movimento espírita é um esforço organizado para reunir pessoas e recursos, de maneira estruturada, para promover o estudo, a vivência e a divulgação do Espiritismo. A estrutura do movimento espírita brasileiro é de natureza federativa e foi desenhada através de um acordo voluntário \textit{ad referendum} que entre si fizeram alguns representantes espíritas quando em 1949 assinaram conjuntamente o documento denominado Pacto Áureo, do qual resultou o Conselho Federativo Nacional (CFN). 

Apesar das críticas existentes sobre o documento e sua origem, identificamos um desenho da estrutura do Movimento Espírita Brasileiro que tem prevalecido e, especialmente no item 12, temos que:


\begin{citacao}
As Sociedades componentes do Conselho Federativo Nacional são completamente independentes. A ação do Conselho só se verificará, aliás, fraternalmente, no caso de alguma Sociedade passar a adotar programa que colida com a doutrina exposta nas obras: “O Livro dos Espíritos” e “O Livro dos Médiuns”, e isso por ser ele, o Conselho, o orientador do Espiritismo no Brasil.
\end{citacao}

Deste texto destacamos a natureza independente, completamente independente das sociedades que o compõe o CFN e a natureza fraterna, portanto em condições de completa igualdade, para o caso de alguma Sociedade passar a adotar programa que colida com a doutrina exposta nas obras: “O Livro dos Espíritos” e “O Livro dos Médiuns”.

Engana-se quem julgar que a estrutura do CFN possui qualquer hierarquia sobre as sociedades espíritas ou sobre qualquer indivíduo. Além de compreender que a lei de liberdade é um dos fundamentos do pensamento espírita que não pode haver progresso sem o natural exercício da liberdade, o CFN não caracteriza-se como uma instância hierárquica para o movimento espírita brasileiro, mas como um fórum democrático para a apreciação de temas de interesse do movimento espírita brasileiro e para planejar e executar ações coletivas de estudo, vivência e divulgação do espiritismo sobre as bases da codificação kardequiana. 

Não há Espiritismo sem o respeito à Lei de Progresso \cite[Parte III. Cap. VII ]{Kardec1857} e à Lei do Liberdade \cite[Parte III. Cap.X]{Kardec1857}, ambas caracterizadas na Filosofia Espírita como leis morais que regulam o universo do espírito como as leis físicas regulam o universo da matéria. O Espiritismo é progressista em sua natureza. Ao definir o espírito como ente natural e sujeito ao contínuo aprendizado das leis divinas através das múltiplas experiências que o Universo lhe proporciona, a doutrina espírita assinala que todos os seres estão em contínua interação, cujo propósito é o progresso -- aqui entendido como a contínua aproximação de um estado de melhoria em concordância com as Leis Divinas (que são as leis naturais). 

Por sua vez, no atual estágio de progresso, o movimento espírita utiliza o processo democrático representativo para analisar temas e para tomar decisões no âmbito do CFN. Nele todos os estados estão representados por suas federativas, que por sua vez intentam representar as instituições espíritas de suas localidades. Mas as decisões tomadas não têm caráter impositivo. Tratam-se de recomendações ou orientações cuja adesão é sempre voluntária, sempre baseada no respeito à liberdade e autonomia de cada sociedade ou indivíduo. Neste aspecto o movimento espírita não impõe as decisões com base na vontade da maioria. Planeja, analisa e orienta conforme a decisão da maioria, mas a adoção desta ou daquela plataforma, deste ou daquele modelo de ação, desta ou daquela atitude é sempre uma decisão da sociedade espírita ou do indivíduo, como recomenda a lição de autonomia exarada no Pacto Áureo, em seu item 12.

Ao respeitar o livre-arbítrio das sociedades e dos indivíduos, o movimento espírita evita a ditadura da democracia, cuja imposição da vontade da maioria, pode obscurecer o direito de liberdade do indivíduo. Compreendendo que cada sociedade espírita deve encontrar o seu caminho de progresso, o movimento espírita recomenda, orienta e informa, sem impor, obrigar ou exigir. 

Por esta natureza "ultra"\ democrática do movimento espírita, não se pode esperar responsabilidades hierárquicas. O Conselho Federativo Nacional não é o responsável pelas ações dos espíritas brasileiros, dada a completa autonomia de cada sociedade e indivíduo. A Federação Espírita Brasileira, como entidade que preside o CFN, e nele não tem voto! - é a sociedade responsável por reunir os representantes do movimento espírita e promover o debate de temas, a análise de assuntos que interessem ao movimento espírita brasileiro e coordena as ações para a execução em âmbito nacional, das recomendações decididas pelo CFN, sempre respeitando a autonomia das sociedades e a liberdade de escolha dos indivíduos. 

Neste ano em que se comemoram os 70 anos do Pacto Áureo, mais do que refletirmos sobre a sua natureza, é necessário analisar o valor da estrutura proposta e o quanto, voluntariamente, podemos ganhar com a adoção de suas propostas. Desde a sua constituição, modificações foram propostas para o funcionamento do Conselho Federativo Nacional, que se refletem em seu Regimento Interno, mas a essência democrática de sua estrutura, o completo respeito à autonomia das sociedades, tanto quanto  à liberdade dos indivíduos,  mantêm o compromisso com o progresso dos espíritos e com a melhoria da sociedade.

O documento original, posteriormente denominado Pacto Áureo, é simplesmente a ata de uma reunião \cite{pactoaureo}. Destacado o caráter ad referendum de sua proposta - que significa ``sujeito à aceitação posterior por parte de um colegiado (diz-se de ato tomado isoladamente)'', é necessário compreendermos que não se trata de uma hierarquia à qual devemos obediência, mas um programa de trabalho democraticamente constituído e ao qual podemos, ou não, aderir, conforme nossas convicções e interesses em relação ao movimento espírita brasileiro. Sua proposta continua em concordância com o caráter de desenvolvimento de trabalhos coletivos proposto por Allan Kardec: Trabalho, Tolerância e Solidariedade. E qualificamos: democrática e voluntariamente.  

\vspace{1,2cm}
\hspace{\fill} Brasília, \imprimirdata.

% ----------------------------------------------------------
% ELEMENTOS PÓS-TEXTUAIS
% ----------------------------------------------------------
\postextual

% ----------------------------------------------------------
% Referências bibliográficas
% ----------------------------------------------------------
\bibliography{espiritas,referencias}



\end{document}
